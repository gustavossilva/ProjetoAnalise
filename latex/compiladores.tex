\documentclass[12pt,a4paper,twoside]{report}
% -------------------------------------------------------------------- %
% Pacotes

\usepackage[utf8]{inputenc}
\usepackage[T1]{fontenc}
\usepackage[brazil]{babel}
\usepackage[fixlanguage]{babelbib}
\usepackage[pdftex]{graphicx}      % usamos arquivos pdf/png como figuras
\usepackage{setspace}              % espaçamento flexvel
\usepackage{indentfirst}           % indentação do primeiro parágrafo
\usepackage{makeidx}               % índice remissivo
\usepackage[nottoc]{tocbibind}     % acrescentamos a bibliografia/indice/conteudo no Table of Contents
\usepackage{courier}               % usa o Adobe Courier no lugar de Computer Modern Typewriter
\usepackage{type1cm}               % fontes realmente escaláveis
\usepackage{titletoc}
\usepackage{ucs}
\usepackage[font=small,format=plain,labelfont=bf,up,textfont=it,up]{caption}
\usepackage[usenames,svgnames,dvipsnames]{xcolor}
\usepackage[a4paper,top=2.54cm,bottom=2.0cm,left=2.0cm,right=2.54cm]{geometry} % margens
\usepackage{amsmath} 
\usepackage{float}

\usepackage[pdftex,plainpages=false,pdfpagelabels,pagebackref,colorlinks=true,citecolor=DarkGreen,
linkcolor=NavyBlue,urlcolor=DarkRed,filecolor=green,bookmarksopen=true]{hyperref} % links coloridos
\usepackage[all]{hypcap}                % soluciona o problema com o hyperref e capítulos
\usepackage[square,sort,nonamebreak,comma]{natbib}  % citação bibliográfica alpha
\fontsize{60}{62}\usefont{OT1}{cmr}{m}{n}{\selectfont}
\usepackage{upquote}                    % formata apóstrofes '
\usepackage{textcomp}

% Para formatar corretamente as URLs
\usepackage{url}
% -------------------------------------------------------------------- %
% Cabeçalhos similares ao TAOCP de Donald E. Knuth
\usepackage{fancyhdr}
\pagestyle{fancy}
\fancyhf{}
\renewcommand{\chaptermark}[1]{\markboth{\MakeUppercase{#1}}{}}
\renewcommand{\sectionmark}[1]{\markright{\MakeUppercase{#1}}{}}
\renewcommand{\headrulewidth}{0pt}

% -------------------------------------------------------------------- %
\graphicspath{{./imagens/}}        % caminho das figuras
\frenchspacing                     % arruma o espaço: id est (i.e.) e exempli gratia (e.g.)
\urlstyle{same}                    % URL com o mesmo estilo do texto e no mono-spaced
\makeindex                         % para o índice remissivo
\raggedbottom                      % para no permitir espaços extras no texto
\fontsize{60}{62}\usefont{OT1}{cmr}{m}{n}{\selectfont}
\cleardoublepage
\normalsize

% -------------------------------------------------------------------- %
% Cores para formatação de código
\usepackage{color}
\definecolor{vermelho}{rgb}{0.6,0,0} % para strings
\definecolor{verde}{rgb}{0.25,0.5,0.35} % para comentários
\definecolor{roxo}{rgb}{0.5,0,0.35} % para palavras-chaves
\definecolor{azul}{rgb}{0.25,0.35,0.75} % para strings
\definecolor{cinza-claro}{gray}{0.95}
% -------------------------------------------------------------------- %
% Opções de listagem usados para o código fonte
% Ref: http://en.wikibooks.org/wiki/LaTeX/Packages/Listings



\usepackage{listings}           % para formatar código-fonte (ex. em Java)


\lstset{ %
language=[Objective]Caml,  % seleciona a linguagem do código (aqui em lstlang0.sty
basicstyle=\footnotesize\ttfamily, % o tamanho da fonte usado no código
commentstyle=\color{verde}\bfseries,  % formatação de comentários
stringstyle=\color{azul},    % formatação de strings
upquote=true,
numbers=left,                   % onde colocar os números de linha
numberstyle=\tiny,  % o tamanho da fonte usada para a numeração das linhas
stepnumber=1,                   % o intervalo entre dois números de linhas. Se for 1, numera cada uma.
numbersep=5pt,                  % how far the line-numbers are from the code
showspaces=false,               % show spaces adding particular underscores
showstringspaces=false,         % underline spaces within strings
showtabs=false,                 % show tabs within strings adding particular underscores
keywordstyle=\color{roxo}\bfseries,
keywordstyle=[1]\color{roxo}\bfseries,
keywordstyle=[2]\color{verde}\bfseries,
%        keywordstyle=[3]\textbf,    %
%        keywordstyle=[4]\textbf,   \sqrt{\sqrt{}} %
frame=b,                   % adds a frame around the code
framerule=0.6pt,
tabsize=2,                      % sets default tabsize to 2 spaces
captionpos=t,                   % sets the caption-position to top
breaklines=true,                % sets automatic line breaking
breakatwhitespace=false,        % sets if automatic breaks should only happen at whitespace
escapeinside={\%*}{*)},         % if you want to add a comment within your code
backgroundcolor=\color[rgb]{1.0,1.0,1.0}, % choose the background color.
rulecolor=\color[rgb]{0.8,0.8,0.8},
extendedchars=true,
xleftmargin=10pt,
xrightmargin=10pt,
framexleftmargin=10pt,
framexrightmargin=10pt,
literate={â}{{\^{a}}}1  % para formatar corretamente os acentos do Português ao usar utf8
    {ê}{{\^{e}}}1
    {ô}{{\^{o}}}1  
    {Â}{{\^{A}}}1
    {Ê}{{\^{E}}}1
    {Ô}{{\^{O}}}1
    {á}{{\'{a}}}1
    {é}{{\'{e}}}1
    {í}{{\'{i}}}1
    {ó}{{\'{o}}}1
    {ú}{{\'{u}}}1
    {Á}{{\'{A}}}1
    {É}{{\'{E}}}1
    {Í}{{\'{I}}}1
    {Ó}{{\'{O}}}1
    {Ú}{{\'{U}}}1
    {à}{{\`{a}}}1
    {À}{{\`{A}}}1
    {ã}{{\~{a}}}1
    {õ}{{\~{o}}}1
    {Ã}{{\~{A}}}1
    {Õ}{{\~{O}}}1
    {ç}{{\c{c}}}1
    {Ç}{{\c{C}}}1
    {ü}{{\"u}}1
    {Ü}{{\"U}}1
}

\renewcommand{\lstlistingname}{Listagem}
\renewcommand{\lstlistlistingname}{Lista de Listagens}

% Definição de novos estilos
\lstdefinestyle{Bash}
    {language=bash,frame=single,numbers=none,basicstyle=\footnotesize\ttfamily,
     morekeywords={cp,mkdir,sudo,tar}}

% Definição de novos ambientes
\lstnewenvironment{terminal}
  {\lstset{style=Bash}}
  {}

\lstnewenvironment{ocaml}
  {\lstset{basicstyle=\scriptsize\ttfamily,
           frame=single,
           frameround=tttt,
           framerule=2pt,
           numbers=none,
           rulecolor=\color{Salmon}}}
  {}

\lstnewenvironment{xml}
   {\lstset{language=XML,frame=single,numbers=none}}
   {}

\lstnewenvironment{interprete}
  {\lstset{frame=single,
            frameround=tttt,
            numbers=none,
            basicstyle=\ttfamily,
            framerule=2pt,
            rulecolor=\color{CadetBlue}}}
  {}
% Formata o caption da listagem
% \DeclareCaptionFont{blue}{\color{blue}} 

% \captionsetup[lstlisting]{singlelinecheck=false, labelfont={blue}, textfont={blue}}
\usepackage{caption}
\DeclareCaptionFont{white}{\color{white}}
\DeclareCaptionFormat{listing}{\colorbox[cmyk]{0.43, 0.35, 0.35,0.01}{\parbox{\textwidth}{\hspace{15pt}#1#2#3}}}
\captionsetup[lstlisting]{format=listing,labelfont=white,textfont=white, singlelinecheck=false, margin=0pt, font={bf,footnotesize}}

\newcommand{\ListingsPath}{./codigos}
% Inclui o nome do arquivo como Caption 
\newcommand{\filelisting}[2][]{%
    \lstinputlisting[caption={\texttt{\detokenize{#2}}},#1]{\ListingsPath/#2}%
}

% ---------------------------------------------------------------------------- %

% ---------------------------------------------------------------------------- %

\title{Análise de Algoritmos - Ordenação}
\date{}
\author{Gustavo de Souza Silva \\ Guilherme de Souza Silva \\ Arthur Xavier \\ Schumaiquer Souto \\
\vspace{1cm} \\
Faculdade de Computação \\
Universidade Federal de Uberlândia
}
\date{\today}

%\includeonly{cap-clojure,magical,short}
\begin{document}
  \maketitle
% -------------------------------------------------------------------- %
% Listas de figuras, tabelas e códigos criadas automaticamente
\listoffigures            
\listoftables            
\lstlistoflistings
% -------------------------------------------------------------------- %

% -------------------------------------------------------------------- %
% Sumário
\tableofcontents    

% Capítulos do trabalho

% cabeçalho para as páginas de todos os capítulos
\fancyhead[RE,LO]{\thesection}

%\singlespacing              % espaçamento simples
\setlength{\parskip}{0.15in} % espaçamento entre paragráfos

\chapter{Introdução}
Este relatŕio tem como objetivo fazer a análise de diversos algoritmos já conhecidos de ordenação. O intúito desde trabalho é comprovar que as provas matemáticas realmente acontecem em um ambiente real de execução.

\section{Codificação}
O arquivo vetor.c mantém todas as funções a respeito do vetor, como geração, preenchimento, etc.
\lstinputlisting[label={arq:vetor.c}, language=C, caption={Arquivo referente ao vetor}]{../vetor.c}

Este arquivo serve para gerar os vetores e salva-los em arquivos.
\lstinputlisting[label={arq:gera_vets.c}, language=C, caption={Geração dos vetores}]{../gera_vets.c}

Este arquivo contém os algoritmos de ordenação pedidos.
\lstinputlisting[label={arq:ordena.c}, language=C, caption={Métodos de ordenação}]{../ordena.c}

O arquivo ensaios.c serve para automatizar e calcular os tempos de cada método de ordenação.
\lstinputlisting[label={arq:ensaios.c}, language=C, caption={Automatização dos experimentos}]{../ensaios.c}

\subsection{Comandos}
Os seguintes passos devem ser seguidos para criação dos vetores que serão utilizados no experimento:
1 - Compilar o arquivo vetor.c;
\begin{terminal}
    > gcc -O3 -c vetor.c
\end{terminal}
2 - Compilar o programar que gera os vetores e os coloca no diretório determinado;
\begin{terminal}
    > gcc -O3 vetor.o gera_vets.c -o gera_vets.exe
\end{terminal}
3 - Para usá-lo digite
\begin{terminal}
    > ./gera_vets.exe
\end{terminal}

Os passos a seguir são para execução do experimento>
1 - Verifique a existência do diretório contendo os vetores, e então digite o seguinte comando:
\begin{terminal}
    > gcc -O3 -c ordena.c
\end{terminal}
2 - Agora é necessário compilar o arquivo de ensaio e tudo que será utilizado
\begin{terminal}
    > gcc -O3 vetor.o ordena.o ensaios.c -o ensaios.exe
\end{terminal}
3 - Para executar digite:
\begin{terminal}
    > ./ensaios.exe
\end{terminal}

\chapter{Gráficos de Funções Matemáticas}
Colocar os gráficos de n nlogn e talz.

\chapter{Insertion Sort}
Texto sobre o insertion

\section{Insertion Sort - Vetor Aleatório}
Um pequeno texto falando sobre o vetor totalmente aleatório e o insertion.
\begin{table}[H]
\centering
\caption{Insertion Sort com Vetor aleatório}
\begin{tabular}{|l|l|}
\hline
\multicolumn{1}{|c|}{\textbf{Número de Elementos}} & \multicolumn{1}{c|}{\textbf{Tempo de execução em nanosegundos}} \\ \hline
16 & 592 \\ \hline
32 & 623 \\ \hline
64 & 1330 \\ \hline
128 & 3921 \\ \hline
256 & 13475 \\ \hline
512 & 49717 \\ \hline
1024 & 181720 \\ \hline
2048 & 709142 \\ \hline
4096 & 2818906 \\ \hline
8192 & 11332358 \\ \hline
16384 & 44220895 \\ \hline
\end{tabular}
\end{table}


\subsection{Gráfico Insertion sort - Vetor Aletório}
\begin{figure}[H]
    \centering
    \includegraphics[width=0.7\linewidth]{graficos/Insertion/vIntAleatorio/vIntAleatorio.png}
  \caption{Gráfico Insertion Sort - Vetor Aleatorio}
\end{figure}


\section{Insertion Sort - Vetor Crescente}
Pequeno texto sobre o vetor

\begin{table}[H]
\centering
\caption{Insertion Sort com Vetor ordenado em ordem crescente}
\label{my-label}
\begin{tabular}{|l|l|}
\hline
\multicolumn{1}{|c|}{\textbf{Número de Elementos}} & \multicolumn{1}{c|}{\textbf{Tempo de execução em nanosegundos}} \\ \hline
16 & 330 \\ \hline
32 & 359 \\ \hline
64 & 366 \\ \hline
128 & 441 \\ \hline
256 & 653 \\ \hline
512 & 1151 \\ \hline
1024 & 1616 \\ \hline
2048 & 3006 \\ \hline
4096 & 5551 \\ \hline
8192 & 11105 \\ \hline
16384 & 21993 \\ \hline
\end{tabular}
\end{table}

\subsection{Grafico Insertio sort - Vetor Crescente}
\begin{figure}[H]
    \centering
    \includegraphics[width=0.7\linewidth]{graficos/Insertion/vIntCrescente/vIntCrescente.png}
  \caption{Gráfico Insertion Sort - Vetor Crescente}
\end{figure}
\chapter{Merge Sort}

%para ajuda
%\lstinputlisting[label={arq:prog1.c}, language=C, caption={Módulo Mínimo C}]{codigos/MiniC/prog1.c}
%\lstinputlisting[label={arq:prog4.c}, language=C, caption={Atribuição de uma soma de inteiros a uma variável C}]{codigos/MiniC/prog4.c}
%\begin{terminal}
%> sudo apt-get install llvm
%> sudo apt-get install clang
%\end{terminal}

%\chapter{Referências}
%\href{http://llvm.org/releases/3.3/docs/LangRef.html}{LLVM Documentation}\\
%\href{https://groups.google.com/forum/#!forum/comp_ufu}{Trabalhos de outros alunos do curso}\\
%  
\end{document} 
